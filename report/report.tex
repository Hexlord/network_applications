\documentclass[a4paper,12pt]{extarticle}
\usepackage[utf8x]{inputenc}
\usepackage[T1,T2A]{fontenc}
\usepackage[russian]{babel}
\usepackage{hyperref}
\usepackage{indentfirst}
\usepackage{listings}
\usepackage{color}
\usepackage{here}
\usepackage{array}
\usepackage{multirow}
\usepackage{graphicx}

\usepackage{caption}
\renewcommand{\lstlistingname}{Программа} % заголовок листингов кода

\bibliographystyle{ugost2008ls}

\usepackage{listings}
\lstset{ %
extendedchars=\true,
keepspaces=true,
language=C,						% choose the language of the code
basicstyle=\footnotesize,		% the size of the fonts that are used for the code
numbers=left,					% where to put the line-numbers
numberstyle=\footnotesize,		% the size of the fonts that are used for the line-numbers
stepnumber=1,					% the step between two line-numbers. If it is 1 each line will be numbered
numbersep=5pt,					% how far the line-numbers are from the code
backgroundcolor=\color{white},	% choose the background color. You must add \usepackage{color}
showspaces=false				% show spaces adding particular underscores
showstringspaces=false,			% underline spaces within strings
showtabs=false,					% show tabs within strings adding particular underscores
frame=single,           		% adds a frame around the code
tabsize=2,						% sets default tabsize to 2 spaces
captionpos=t,					% sets the caption-position to top
breaklines=true,				% sets automatic line breaking
breakatwhitespace=false,		% sets if automatic breaks should only happen at whitespace
escapeinside={\%*}{*)},			% if you want to add a comment within your code
postbreak=\raisebox{0ex}[0ex][0ex]{\ensuremath{\color{red}\hookrightarrow\space}},
texcl=true,
inputpath=listings,                     % директория с листингами
}

\usepackage[left=2cm,right=2cm,
top=2cm,bottom=2cm,bindingoffset=0cm]{geometry}

%% Нумерация картинок по секциям
\usepackage{chngcntr}
\counterwithin{figure}{section}
\counterwithin{table}{section}

%%Точки нумерации заголовков
\usepackage{titlesec}
\titlelabel{\thetitle.\quad}
\usepackage[dotinlabels]{titletoc}

%% Оформления подписи рисунка
\addto\captionsrussian{\renewcommand{\figurename}{Рисунок}}
\captionsetup[figure]{labelsep = period}

%% Подпись таблицы
\DeclareCaptionFormat{hfillstart}{\hfill#1#2#3\par}
\captionsetup[table]{format=hfillstart,labelsep=newline,justification=centering,skip=-10pt,textfont=bf}

%% Путь к каталогу с рисунками
\graphicspath{{fig/}}


\begin{document}	% начало документа

% Титульная страница
\begin{titlepage}	% начало титульной страницы

	\begin{center}		% выравнивание по центру

		\large Санкт-Петербургский политехнический университет Петра Великого\\
		\large Институт компьютерных наук и технологий \\
		\large Кафедра компьютерных систем и программных технологий\\[6cm]
		% название института, затем отступ 6см
		
		\huge Базы данных\\[0.5cm] % название работы, затем отступ 0,5см
		\large Отчет по лабораторной работе №1\\[0.1cm]
		\large "Разработка структуры БД"\\[5cm]

	\end{center}


	\begin{flushright} % выравнивание по правому краю
		\begin{minipage}{0.25\textwidth} % врезка в половину ширины текста
			\begin{flushleft} % выровнять её содержимое по левому краю

				\large\textbf{Работу выполнил:}\\
				\large Кнорре А.В.\\
				\large {Группа:} 43501/3\\
				
				\large \textbf{Преподаватель:}\\
				\large Мяснов А.В.

			\end{flushleft}
		\end{minipage}
	\end{flushright}
	
	\vfill % заполнить всё доступное ниже пространство

	\begin{center}
	\large Санкт-Петербург\\
	\large \the\year % вывести дату
	\end{center} % закончить выравнивание по центру

\thispagestyle{empty} % не нумеровать страницу
\end{titlepage} % конец титульной страницы

\vfill % заполнить всё доступное ниже пространство


\pagestyle{empty} 
% Содержание
\pagenumbering{arabic}
\renewcommand\contentsname{\centerline{Содержание}}
\tableofcontents
\newpage 
\pagestyle{plain} 
\setcounter{page}{3}


\section{Цель работы}
Познакомиться с основами проектирования схемы БД, способами
организации данных в SQL-БД.

\section{Программа работы}
\begin{itemize}
\item Создание проекта для работы в GitLab.
\item Выбор задания (предметной области), описание набора данных и
требований к хранимым данным в свободном формате в wiki своего
проекта в GitLab.
\item Формирование в свободном формате (предпочтительно в виде
графической схемы) cхемы БД, соответствующей заданию. Должно
получиться не менее 7 таблиц.
\item Согласование с преподавателем схемы БД. Обоснование принятых
решений и соответствия требованиям выбранного задания.
\item Выкладывание схемы БД в свой проект в GitLab.
\item Демонстрация результатов преподавателю.
\end{itemize}

\section{Ход выполнения работы}
\subsection{Задание}
\begin{itemize}
\item Проект в GitLab: \url{http://gitlab.icc.spbstu.ru/Hexlord/galaxy_db}
\item Название: Галактическая торговая коалиция
\end{itemize}
\subsection{Содержание БД}
\begin{enumerate}

\item {\it system} -- содержит список солнечных систем
\begin{enumerate}
\item {\it name} -- название солнечной системы
\end{enumerate}

\item {\it planet} -- содержит список планет
\begin{enumerate}
\item {\it name} -- название планеты
\item {\it system\_id} -- номер солнечной системы
\end{enumerate}

\item {\it crew} -- содержит список членов команды
\begin{enumerate}
\item {\it name} -- имя члена команды
\item {\it ship\_id} -- номер корабля
\end{enumerate}

\item {\it ship\_type} -- содержит список типов кораблей
\begin{enumerate}
\item {\it name} -- название типа корабля
\item {\it speed} -- максимальная тяга корабля в g
\end{enumerate}

\item {\it ship} -- содержит список кораблей
\begin{enumerate}
\item {\it pilot\_id} -- номер пилота
\item {\it ship\_type\_id} -- номер типа корабля
\end{enumerate}

\item {\it planet\_ship} -- содержит список связей между планетами и кораблями, которые совершили на них посадку
\begin{enumerate}
\item {\it planet\_id} -- номер планеты
\item {\it ship\_id} -- номер корабля
\end{enumerate}

\item {\it item\_type} -- содержит список типов предметов
\begin{enumerate}
\item {\it name} -- название предмета
\item {\it description} -- описание предмета
\end{enumerate}

\item {\it item} -- содержит список предметов
\begin{enumerate}
\item {\it item\_type\_id} -- номер типа предмета
\item {\it quantity} -- количество предметов
\end{enumerate}

\item {\it ship\_item} -- содержит список связей между кораблями и их предметами
\begin{enumerate}
\item {\it ship\_id} -- номер корабля
\item {\it item\_id} -- номер предмета
\end{enumerate}

\end{enumerate}

\subsection{Схема}

\begin{figure}[H]
	\begin{center}
  		\makebox[\textwidth]{\includegraphics[scale=0.6]{../../galaxy_db.png}}
		\caption{Созданная схема galaxy} 
		\label{pic:scheme_galaxy} % название для ссылок внутри кода
	\end{center}
\end{figure}

\section{Выводы}
Удалось спроектировать схему базы данных, которая удовлетворяет постановленному заданию. Для этого были созданы таблицы с необходимыми ограничениями для контроля за качеством входящих данных. Использованы списки связей типа один-многие, что было предложено преподавателем. Любые категории данных представлены в виде записей в таблицах, то есть система максимально расширяема. Данные не дублируются, аномалий нет.
\end{document}
