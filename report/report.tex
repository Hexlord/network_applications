\include{settings}

\begin{document}	% начало документа

% Титульная страница
\include{titlepage}

\pagestyle{empty} 
% Содержание
\pagenumbering{arabic}
\renewcommand\contentsname{\centerline{Содержание}}
\tableofcontents
\newpage 
\pagestyle{plain} 
\setcounter{page}{3}


\section{Цель работы}
Познакомиться с основами проектирования схемы БД, способами
организации данных в SQL-БД.

\section{Программа работы}
\begin{itemize}
\item Создание проекта для работы в GitLab.
\item Выбор задания (предметной области), описание набора данных и
требований к хранимым данным в свободном формате в wiki своего
проекта в GitLab.
\item Формирование в свободном формате (предпочтительно в виде
графической схемы) cхемы БД, соответствующей заданию. Должно
получиться не менее 7 таблиц.
\item Согласование с преподавателем схемы БД. Обоснование принятых
решений и соответствия требованиям выбранного задания.
\item Выкладывание схемы БД в свой проект в GitLab.
\item Демонстрация результатов преподавателю.
\end{itemize}

\section{Ход выполнения работы}
\subsection{Задание}
\begin{itemize}
\item Проект в GitLab: \url{http://gitlab.icc.spbstu.ru/Hexlord/galaxy_db}
\item Название: Галактическая торговая коалиция
\end{itemize}
\subsection{Содержание БД}
\begin{enumerate}

\item {\it system} -- содержит список солнечных систем
\begin{enumerate}
\item {\it name} -- название солнечной системы
\end{enumerate}

\item {\it planet} -- содержит список планет
\begin{enumerate}
\item {\it name} -- название планеты
\item {\it system\_id} -- номер солнечной системы
\end{enumerate}

\item {\it crew} -- содержит список членов команды
\begin{enumerate}
\item {\it name} -- имя члена команды
\item {\it ship\_id} -- номер корабля
\end{enumerate}

\item {\it ship\_type} -- содержит список типов кораблей
\begin{enumerate}
\item {\it name} -- название типа корабля
\item {\it speed} -- максимальная тяга корабля в g
\end{enumerate}

\item {\it ship} -- содержит список кораблей
\begin{enumerate}
\item {\it pilot\_id} -- номер пилота
\item {\it ship\_type\_id} -- номер типа корабля
\end{enumerate}

\item {\it planet\_ship} -- содержит список связей между планетами и кораблями, которые совершили на них посадку
\begin{enumerate}
\item {\it planet\_id} -- номер планеты
\item {\it ship\_id} -- номер корабля
\end{enumerate}

\item {\it item\_type} -- содержит список типов предметов
\begin{enumerate}
\item {\it name} -- название предмета
\item {\it description} -- описание предмета
\end{enumerate}

\item {\it item} -- содержит список предметов
\begin{enumerate}
\item {\it item\_type\_id} -- номер типа предмета
\item {\it quantity} -- количество предметов
\end{enumerate}

\item {\it ship\_item} -- содержит список связей между кораблями и их предметами
\begin{enumerate}
\item {\it ship\_id} -- номер корабля
\item {\it item\_id} -- номер предмета
\end{enumerate}

\end{enumerate}

\subsection{Схема}

\begin{figure}[H]
	\begin{center}
  		\makebox[\textwidth]{\includegraphics[scale=0.6]{../../galaxy_db.png}}
		\caption{Созданная схема galaxy} 
		\label{pic:scheme_galaxy} % название для ссылок внутри кода
	\end{center}
\end{figure}

\section{Выводы}
Удалось спроектировать схему базы данных, которая удовлетворяет постановленному заданию. Для этого были созданы таблицы с необходимыми ограничениями для контроля за качеством входящих данных. Использованы списки связей типа один-многие, что было предложено преподавателем. Любые категории данных представлены в виде записей в таблицах, то есть система максимально расширяема. Данные не дублируются, аномалий нет.
\end{document}
